% template for producing IEEE-format articles using LaTeX2e.
% written by Matthew Ward, CS Department, Worcester Polytechnic Institute.
% adapted to LaTeX2e and latex8.sty by Felix Gartner, TUD.
%
% This file can be used for producing german articles.
%
\documentclass[10pt,twocolumn]{article}
\usepackage{latex8}% use IEEE document style
\usepackage{times}% use Postscript fonts
\usepackage{epsfig}% enable EPS figures
\usepackage{german}% German language
\pagestyle{empty}% omit page numbers (in final version)
%
\begin{document}


\title{Mein toller Artikel im IEEE-Format}

%for single author 
\author{E. I. N. Author \\
  Mein Fachbereich \\
  Meine Uni \\
  Meine Stadt}
 
%for two authors (just remove % characters)
%\author{\begin{tabular}[t]{c@{\extracolsep{8em}}c}
%  I. M. Author	& M. Y. Coauthor \\
% \\
%  My Department & Coauthor Department \\
%  My Institute & Coauthor Institute \\
%  City, ST~~zipcode	& City, ST~~zipcode
%\end{tabular}}

\maketitle

\thispagestyle{empty}% omit page number on first page (in final version)



\begin{abstract}
  
  Dies ist die Kurzzusammenfassung (der ``abstract''). Es ist
  ein Absatz von etwa 100 bis 150 W"ortern und wird normalerweise
  kursiv gesetzt. Er enth"alt den Inhalt der Arbeit in Kurzform.

\end{abstract}

% latex8.sty uses \Section, \SubSection instead of \section, \subsection
\Section{Einleitung}

Dies ist die Einleitung. Hier wird ein wenig dar"uber erz"ahlt, was man
beschreiben will und warum das wichtig und interessant ist.

\SubSection{Verwandte Arbeiten}

Dies ist ein Unterabschnitt. Man kann generell auch Eintr"age im
Literaturverzeichnis referenzieren. Zum Beispiel die Arbeit
\cite{key:foo} oder \cite{foo:baz}. \LaTeX\ sorgt daf"ur, da"s die
Nummern richtig im Text erscheinen.  Die Nummern basieren auf der
Reihenfolge, in der die Eintr"age im Literaturverzeichnis stehen.

\LaTeX\ setzt eckicke Klammern um die Referenzzahlen. Falls 
statt einer Referenzzahl pl"otzlich Fragezeichen (``??'') im Text
erscheinen, mu"s man \LaTeX\ zweimal "uber den Text r"uberlaufen
lassen.

\Section{Hauptteil}
\label{sec:main}

Dies ist ein normaler Abschnitt, der den Hauptteil des Textes
enthalten kann. Man kann auch Abbildungen benutzen, die als
EPS-Dateien vorliegen m"ussen. Man kann Abbildungen und Abschnitte
auch referenzieren, zum Beispiel die Abbildung \ref{fig:sample} in
Abschnitt \ref{sec:main}.

\begin{figure}
  \begin{center}
    \leavevmode
    % this is where the EPS file is included:
    \epsfig{file=sample.eps,width=6cm}
    \caption{Eine Beispielabbildung.}
    \label{fig:sample}
  \end{center}
\end{figure}


\Section{Zusammenfassung}

Dieser Rahmen wird vermutlich ausreichend sein, um die ersten
Klippen von \LaTeX\ zu umschiffen. Bei der weiteren Arbeit w"unschen
wir viel Gl"uck!


%this is how to do an unnumbered subsection
\subsection*{Danksagungen}

So kann man einen unnumerierten Absatz erzeugen. In ihm kann man
zum Beispiel Danksagungen unterbringen.



\bibliographystyle{latex8}
\begin{thebibliography}{9}

\bibitem{key:foo}
I. M. Author, 
``Some Related Article I Wrote,''
{\em Some Fine Journal}, Vol. 17, Seiten 1-100, 1987.

\bibitem{foo:baz}
A. N. Expert, 
{\em A Book He Wrote,}
His Publisher, 1989.

\end{thebibliography}

\end{document}

